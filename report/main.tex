\documentclass[sigplan,10pt,review]{acmart}\settopmatter{printfolios=true,printccs=false,printacmref=false}
%%
%% \BibTeX command to typeset BibTeX logo in the docs
\AtBeginDocument{%
  \providecommand\BibTeX{{%
    \normalfont B\kern-0.5em{\scshape i\kern-0.25em b}\kern-0.8em\TeX}}}

%% Rights management information.  This information is sent to you
%% when you complete the rights form.  These commands have SAMPLE
%% values in them; it is your responsibility as an author to replace
%% the commands and values with those provided to you when you
%% complete the rights form.
\setcopyright{acmcopyright}
\copyrightyear{2018}
\acmYear{2018}
\acmDOI{10.1145/1122445.1122456}

%% These commands are for a PROCEEDINGS abstract or paper.
\acmConference[Woodstock '18]{Woodstock '18: ACM Symposium on Neural
  Gaze Detection}{June 03--05, 2018}{Woodstock, NY}
\acmBooktitle{Woodstock '18: ACM Symposium on Neural Gaze Detection,
  June 03--05, 2018, Woodstock, NY}
\acmPrice{15.00}
\acmISBN{978-1-4503-XXXX-X/18/06}


%%
%% Submission ID.
%% Use this when submitting an article to a sponsored event. You'll
%% receive a unique submission ID from the organizers
%% of the event, and this ID should be used as the parameter to this command.
%%\acmSubmissionID{123-A56-BU3}

%%
%% The majority of ACM publications use numbered citations and
%% references.  The command \citestyle{authoryear} switches to the
%% "author year" style.
%%
%% If you are preparing content for an event
%% sponsored by ACM SIGGRAPH, you must use the "author year" style of
%% citations and references.
%% Uncommenting
%% the next command will enable that style.
%%\citestyle{acmauthoryear}

%%
%% \BibTeX command to typeset BibTeX logo in the docs
\AtBeginDocument{%
  \providecommand\BibTeX{{%
    \normalfont B\kern-0.5em{\scshape i\kern-0.25em b}\kern-0.8em\TeX}}}

%% Rights management information.  This information is sent to you
%% when you complete the rights form.  These commands have SAMPLE
%% values in them; it is your responsibility as an author to replace
%% the commands and values with those provided to you when you
%% complete the rights form.
\setcopyright{acmcopyright}
\copyrightyear{2018}
\acmYear{2018}
\acmDOI{10.1145/1122445.1122456}

%% These commands are for a PROCEEDINGS abstract or paper.
\acmConference[Woodstock '18]{Woodstock '18: ACM Symposium on Neural
  Gaze Detection}{June 03--05, 2018}{Woodstock, NY}
\acmBooktitle{Woodstock '18: ACM Symposium on Neural Gaze Detection,
  June 03--05, 2018, Woodstock, NY}
\acmPrice{15.00}
\acmISBN{978-1-4503-XXXX-X/18/06}


%%
%% Submission ID.
%% Use this when submitting an article to a sponsored event. You'll
%% receive a unique submission ID from the organizers
%% of the event, and this ID should be used as the parameter to this command.
%%\acmSubmissionID{123-A56-BU3}

%%
%% The majority of ACM publications use numbered citations and
%% references.  The command \citestyle{authoryear} switches to the
%% "author year" style.
%%
%% If you are preparing content for an event
%% sponsored by ACM SIGGRAPH, you must use the "author year" style of
%% citations and references.
%% Uncommenting
%% the next command will enable that style.
%%\citestyle{acmauthoryear}
%% Bibliography style
\bibliographystyle{ACM-Reference-Format}
%% Citation style
%% Note: author/year citations are required for papers published as an
%% issue of PACMPL.
\citestyle{acmauthoryear}   %% For author/year citations


%%%%%%%%%%%%%%%%%%%%%%%%%%%%%%%%%%%%%%%%%%%%%%%%%%%%%%%%%%%%%%%%%%%%%%
%% Note: Authors migrating a paper from PACMPL format to traditional
%% SIGPLAN proceedings format must update the '\documentclass' and
%% topmatter commands above; see 'acmart-sigplanproc-template.tex'.
%%%%%%%%%%%%%%%%%%%%%%%%%%%%%%%%%%%%%%%%%%%%%%%%%%%%%%%%%%%%%%%%%%%%%%


%% Some recommended packages.
\usepackage{svg}
\usepackage{booktabs}   %% For formal tables:
                        %% http://ctan.org/pkg/booktabs
\usepackage{subcaption} %% For complex figures with subfigures/subcaptions
                        %% http://ctan.org/pkg/subcaption
\usepackage[utf8]{inputenc}
\usepackage{amsmath,amssymb}
\usepackage{amsthm}
\usepackage{algorithm}
\usepackage{algpseudocode}
\usepackage{todonotes}
\usepackage[misc,geometry]{ifsym} 

\usepackage{xspace}
\usepackage{stmaryrd}
\usepackage{comment}

\usepackage{multirow}

\usepackage{amsfonts}
\usepackage{amsmath}
\usepackage{amssymb}
\usepackage{xcolor,colortbl}
\definecolor{babypurple}{RGB}{222,201,255}
\definecolor{lightgoldenrodyellow}{rgb}{0.98, 0.98, 0.82}
\definecolor{babyred}{RGB}{220,210,214}
%\definecolor{babygreen}{RGB}{169,189,183}
\definecolor{camouflagegreen}{rgb}{0.47, 0.53, 0.42}
\definecolor{babygreen}{rgb}{0.47, 0.53, 0.42}
\definecolor{camel}{rgb}{0.76, 0.6, 0.42}
\definecolor{darkpastelpurple}{rgb}{0.59, 0.44, 0.84}
\usepackage{listings}
\usepackage{xparse}
\usepackage{threeparttable}
\usepackage{proof}
\usepackage{enumitem}
\usepackage{ulem}
\normalem

\usepackage{tikz}
\usetikzlibrary{positioning,shapes,arrows}
\usetikzlibrary{patterns}
\usepackage{pgfplots}

\usepackage{booktabs}

\DeclareMathOperator*{\argmax}{arg\,max}
\DeclareMathOperator*{\argmin}{arg\,min}
\newcommand{\toolname}{{\sc Faery}\xspace}
\newcommand{\neo}{{\sc Neo} \xspace}

\newcommand{\semantics}[1]{\llbracket{#1}\rrbracket}
\newcommand{\asemantics}[1]{\llbracket{#1}\rrbracket^{\#}}
\newcommand{\strategy}{\mathcal{S}}

\newcommand{\oracle}[1]{\llbracket{#1}\rrbracket}
\newcommand{\lang}{\mathcal{L}}
\newcommand{\user}{\mathcal{O}}
\newcommand{\mut}{\mu}
%\newcommand{\example}{\mathcal{Q}} this is defined to be something else 
\newcommand{\abs}{\Psi}
\newcommand{\hole}{\square}

\usepackage{xcolor}
\usepackage{bbm}

\usepackage{float}
\newfloat{algorithm}{t}{lop}

\newtheorem{theorem}{Theorem}[section]
\newtheorem{conjecture}[theorem]{Conjecture}
\newtheorem{proposition}[theorem]{Proposition}
\newtheorem{lemma}[theorem]{Lemma}
\newtheorem{corollary}[theorem]{Corollary}
%\newtheorem{example}[theorem]{Example}
\newtheorem{definition}[theorem]{Definition}
\newtheorem{assumption}[theorem]{Assumption}
\newtheorem{sample}[theorem]{Example}



\newcommand{\lattice}{\mathcal{P}}
\newcommand{\element}{P}
\newcommand{\refines}{\sqsubseteq}
\newcommand{\refinedby}{\sqsupseteq}
\newcommand{\aval}{\phi}
\newcommand{\prunes}{\not\models}
\newcommand{\maxprunes}{\not\models^0}
\newcommand{\example}{e}
\newcommand{\examples}{\vec{e}}


\begin{document}

%% Title information
\title[Short Title]{CS292C Final Project Proposal}         %% [Short Title] is optional;
                                        %% when present, will be used in
                                        %% header instead of Full Title.
% \titlenote{with title note}             %% \titlenote is optional;
                                        %% can be repeated if necessary;
                                        %% contents suppressed with 'anonymous'
% \subtitle{Subtitle}                     %% \subtitle is optional
% \subtitlenote{with subtitle note}       %% \subtitlenote is optional;
                                        %% can be repeated if necessary;
                                        %% contents suppressed with 'anonymous'


%% Author information
%% Contents and number of authors suppressed with 'anonymous'.
%% Each author should be introduced by \author, followed by
%% \authornote (optional), \orcid (optional), \affiliation, and
%% \email.
%% An author may have multiple affiliations and/or emails; repeat the
%% appropriate command.
%% Many elements are not rendered, but should be provided for metadata
%% extraction tools.

%% Author with single affiliation.
\author{Alon Albalak}
% \authornote{with author1 note}          %% \authornote is optional;
                                        %% can be repeated if necessary
\orcid{nnnn-nnnn-nnnn-nnnn}             %% \orcid is optional
\affiliation{
  \department{Department of Computer Science}              %% \department is recommended
  \institution{University of California, Santa Barbara}            %% \institution is required
}
\email{alon_albalak@ucsb.edu}          %% \email is recommended

%% Author with two affiliations and emails.
\author{Peter Boyland}
% \authornote{with author2 note}          %% \authornote is optional;
                                        %% can be repeated if necessary
\orcid{nnnn-nnnn-nnnn-nnnn}             %% \orcid is optional
\affiliation{
  \department{Department of Computer Science}             %% \department is recommended
  \institution{University of California, Santa Barbara}           %% \institution is required
}
\email{boyland@ucsb.edu}         %% \email is recommended


%% Abstract
%% Note: \begin{abstract}...\end{abstract} environment must come
%% before \maketitle command
\begin{abstract}
TBD.
\end{abstract}



%% 2012 ACM Computing Classification System (CSS) concepts
%% Generate at 'http://dl.acm.org/ccs/ccs.cfm'.
\begin{CCSXML}
<ccs2012>
<concept>
<concept_id>10011007.10011006.10011008</concept_id>
<concept_desc>Software and its engineering~General programming languages</concept_desc>
<concept_significance>500</concept_significance>
</concept>
<concept>
<concept_id>10003456.10003457.10003521.10003525</concept_id>
<concept_desc>Social and professional topics~History of programming languages</concept_desc>
<concept_significance>300</concept_significance>
</concept>
</ccs2012>
\end{CCSXML}

\ccsdesc[500]{Software and its engineering~General programming languages}
\ccsdesc[300]{Social and professional topics~History of programming languages}
%% End of generated code


%% Keywords
%% comma separated list
\keywords{Program synthesis}  %% \keywords are mandatory in final camera-ready submission


%% \maketitle
%% Note: \maketitle command must come after title commands, author
%% commands, abstract environment, Computing Classification System
%% environment and commands, and keywords command.
\maketitle


\section{Introduction}
\label{sec:intro}

In recent years, there has been an increase of technology used in daily lives, and a subsequent increase in the amount of data generated by the millions of users of such technologies. Not only is data being generated about the users, but internet-connected devices are gathering data on their own environment. In order to understand such data, as well as the underlying systems that the data represents, analysis needs to be done. Manipulating these vast quantities of data then becomes challenging, in part due to the need for skill in a programming language, otherwise, a user must do all their data manipulation manually on a spreadsheet.

Program synthesis is the task of automatically generating a computer program based on some user specifications, and since the 1950s, program synthesis has been the holy grail of computer science~\citep{PGL-010}. In program synthesis, the goal is to create a system that will return valid programs that fulfill a users specifications, when that is possible. When developing a program synthesis framework, there are numerous issues to be overcome. One major obstacle is how the to handle ambiguous user intents. If a users specifications are vague, then there may be many programs which satisfy the specifications, however those specifications may not entirely capture all of a users intent. In these situations, a user must make further amendments to their specifications, and thus the program will become more accurate to the underlying intent of the user. Another challenge is how to reduce calculation times, as there is a limit to how long a user is willing to wait for a solution to their desired specifications.

One solution to these challenges is incremental synthesis. An incremental synthesis system is designed in such a way that if the user gives an ambiguous intent then the system can return a program which fulfills the specifications. However, to handle the ambiguity, the system is able to make adjustments if further specifications are entered by the user to further indicate their intentions. Additionally, if the user does need to make amendments to their specifications, then a program synthesis system will need to run the synthesis over from scratch, whereas an incremental system can simply continue on from the calculations it has already done.
\section{Overview}
\label{sec:overview}

\subsection{Motivating Example}
In this section, we give an overview of our approach to the incremental synthesis problem. We do so by demonstrating the methods on a simple motivating example.

Given the following table \ref{table:1}, and the user specification that they would like to transform the table into the output in table \ref{table:2}, the program synthesis problem attempts to identify table manipulating programs which satisfy the input and output pair, respectively table \ref{table:1} and table \ref{table:2}.

\begin{table}[h]
\centering
\begin{tabular}{|c|c|c|c|c|c|c|}
\hline
ID & g1-0 & g1-10 & g1-30 & g2-0 & g2-10 & g2-30 \\
\hline
A & 1 & 4 & 7 & 10 & 13 & 16 \\
\hline
B & 2 & 5 & 8 & 11 & 14 & 17 \\
\hline
C & 3 & 6 & 9 & 12 & 15 & 18 \\
\hline
\end{tabular}
\caption{Simple example: data}
\label{table:1}
\end{table}

\begin{table}[h]
\centering
\begin{tabular}{|c|c|c|c|}
\hline
A & 0 minutes & 1 & 10  \\
\hline
\end{tabular}
\caption{Simple example: user specification}
\label{table:2}
\end{table}

More specifically, the incremental program synthesis problem attempts to identify satisfying programs without wasting time finding all possible satisfying programs, rather just a subset. In this case, the user has specified that they would like to reformat the table with the first column as the ID. They specify that the second column should be the time derived from the header row, included in the original table as g1-0 for 0 minutes, g1-10 for 10 minutes, etc. They would like the third column to be taken as the value for the given ID, time pair for g1 and the fourth column to be the value for the given ID and time for g2.

As we can see, there are many values in the original table for which the user has not demonstrated their desired transformation. This leads to multiple valid programs and output tables which satisfy the limited specifications. For example, the user may intend for the values from time 0 to be split by ID into separate rows by column, as in table \ref{table:3}. On the other hand, they may actually intend for each time value to be put into a separate row, as in table \ref{table:4}. In the non-incremental program synthesis setting, a program is created which satisfies the user's conditions, but if the program is not correct and the user gives further details, the program synthesis starts from scratch. This extra computation is wasteful, and can be avoided by using an incremental program synthesis setting.

\begin{table}[h]
\centering
\begin{tabular}{|c|c|c|c|}
\hline
A & 0 minutes & 1 & 10 \\
\hline
B & 0 minutes & 2 & 11 \\
\hline
C & 0 minutes & 3 & 12 \\
\hline
\end{tabular}
\caption{Simple example: proposed solution 1}
\label{table:3}
\end{table}

\begin{table}[h]
\centering
\begin{tabular}{|c|c|c|c|}
\hline
A & 0 minutes & 1 & 10 \\
\hline
A & 10 minutes & 4 & 13 \\
\hline
A & 30 minutes & 7 & 16 \\
\hline
\end{tabular}
\caption{Simple example: proposed solution 1}
\label{table:4}
\end{table}

\subsection{Tentative solution}

We aim to improve the incremental program synthesis problem through 2 methods. 

First, if the algorithm determines that a group of programs are likely to be very similar, or have the same properties as designated by their component functions, and a subset of the programs have been determined to satisfy the constraints, then it would be more efficient to search other areas of the search tree to find solutions which are less similar. By grouping similar programs together and caching them for later refinement the algorithm can determine that it need not further explore a particular area of the solution space. In doing so, the search algorithm can more efficiently explore the search space, and spend less time on programs that it already knows can satisfy the given constraints.

Second, using the same methodology as above, the algorithm can deduce those subsets of the search space which may not satisfy the specifications at all. Thus, the search algorithm will spend less time in areas of the search which give partial programs that are unlikely, or impossible, to lead to correct solutions.

Speed is particularly important in incremental synthesis. If the algorithm returns a program which fits the users vague specification, but not their underlying intent, then the user must give further specification and subsequently wait for the algorithm to run. This iterative process may occur multiple times and users are unlikely to be satisfied with long wait times.

As a search heuristic, we will use similarities between programs. For example, we can predict that a program whose family members in the search space are proven to create an output which satisfies the specifications, but which itself cannot be proven to satisfy the input-output pairs, is less likely to be useful to explore. Thus, a search through that section of the search space can be postponed until, and if, the user gives further refinement to their intent.

For the deduction engine in our search, we will be using an SMT solver to test which classes of transformations will be guaranteed to satisfy the given input-output pairs.

\subsection{Challenges}

In the forward-deduction phase of search, there are decisions to be made in balancing the algorithms speed compared with the level of approximation that is acceptable. For example, on tables under some size threshold, it may simply be faster to evaluate the function on the given input-output samples rather than using a deduction engine. Perhaps, it will work better to use some simple algorithm to decide the level of detail in the deduction.

Additionally, we will need to decide which classes of abstract programs to use for the deduction. Some constraints may be more obvious, but others may be more challenging to find yet better performing.

Finally, there is the challenge of programming the deduction itself, either through a theorem prover such as z3 or manually coding a deduction engine for the table manipulation problem.


\section{Problem Formulation}
\label{sec:prob}
TBD.
\section{Interactive Synthesis Algorithm}
\label{sec:algo}

TBD.

\section{Implementation}~\label{sec:impl}
TBD.
\section{Evaluation}~\label{sec:eval}
TBD.
\section{Related work}~\label{sec:related}

In this work, we mostly build off of the work from~\cite{DBLP:journals/pacmpl/WangFBCD20}. They give an algorithm for synthesizing a two-part visualization program from a trace of the output. Their method may run slowly, especially for those desired programs which are deep or with large program classes, but for most of the benchmarks in the human benchmark suite, it takes no more than two demonstrations to get the desired program in the top 10 possibilities.

In \citep{le2017interactive}, the authors provide their solution to interactive program synthesis, which is essentially the same as what we call incremental program synthesis. Besides the difference in the methodology in their paper compared with ours on how to solve under-specified circumstances, another major difference is their focus on string manipulations versus the current paper which focuses on table manipulations.
\section{Conclusion}~\label{sec:concl}
TBD.

%% Bibliography
\bibliography{main}


%% Appendix
% \appendix
% \section{Appendix}

% Text of appendix \ldots

\end{document}
